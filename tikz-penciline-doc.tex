
\documentclass[a4paper]{article}

\usepackage[all]{nowidow}
\widowpenalty10000
\clubpenalty10000


%% See http://tex.stackexchange.com/questions/78334/print-system-information-in-latex-doc
\RequirePackage{ifxetex}
\ifxetex
\RequirePackage{fontspec}
\RequirePackage{xltxtra}
\newcommand{\myxetex}{\XeLaTeX}
\newcommand{\compilator}{\XeLaTeX{}}
\newcommand{\cversion}{\the\XeTeXversion\XeTeXrevision}
\else
\RequirePackage[utf8]{inputenc}
\newcommand{\myxetex}{\texttt{XeLaTeX}}
\newcommand{\compilator}{\LaTeX{}}
\makeatletter
\newcommand{\cversion}{
  \ifnum\pdftexversion<100 %
  \the\pdftexversion.\pdftexrevision
  \else
  \ifnum\pdftexversion<130 %
  \expandafter\@car\the\pdftexversion\@empty\@nil.%
  \expandafter\@cdr\the\pdftexversion\@empty\@nil  
  \pdftexrevision
  \else
  \expandafter\@car\the\pdftexversion\@empty\@nil.%
  \expandafter\@cdr\the\pdftexversion\@empty\@nil.%
  \pdftexrevision
  \fi
  \fi
}
\makeatother
\fi

\makeatletter
\newcommand{\tcbversion}{\tcb@version}
\makeatother

\RequirePackage{etoolbox}
\RequirePackage{minted}

%% http://tex.stackexchange.com/questions/132888/fontawesome-font-not-found
%% https://github.com/wspr/fontspec/issues/165
\defaultfontfeatures{
  Scale=MatchLowercase,
  Path = /usr/local/texlive/2014/texmf-dist/fonts/opentype/public/fontawesome/ }
\RequirePackage{fontawesome}

\RequirePackage{tcolorbox}
\tcbuselibrary{skins,listings}
\tcbuselibrary{minted}
\tcbuselibrary{documentation}

\usepackage{tikz-penciline}
\usetikzlibrary{calc}
\newminted[srcGrid]{latex}{baselinestretch=0.5,fontsize=\footnotesize}


\newtcblisting{exampleB}[2][]{%
  colframe=red!50!yellow!50!black,
  colback=white,
  coltitle=red!50!yellow!3!white,
  bicolor,colbacklower=red!50!yellow!5!white,
  fonttitle=\sffamily\bfseries,
  center upper,
  minted language=latex,
  minted options={baselinestretch=0.5,fontsize=\scriptsize},
  %%lefthand ratio=0.33,
  %%sidebyside,
  text and listing,
  title=#2,#1}

\newtcblisting{exampleA}[2][]{%
  colframe=red!50!yellow!50!black,
  colback=white,
  coltitle=red!50!yellow!3!white,
  bicolor,colbacklower=red!50!yellow!5!white,
  fonttitle=\sffamily\bfseries,
  center upper,
  minted language=latex,
  minted options={baselinestretch=0.5,fontsize=\scriptsize},
  %%lefthand ratio=0.33,
  %%sidebyside,
  listing,
  title=#2,#1}

\newcommand{\PGF}{\textsc{PGF}}
\newcommand{\TikZ}{Ti\textcolor{orange}{\emph{k}}Z}

\title{ \texttt{tikz-penciline}\\ Hand drawing with \PGF/\TikZ\\ }
\author{Sébastien Gross\\
  \faEnvelope ~ \texttt{seb•$\alpha\tau$•chezwam•$\partial\theta\tau$•org}\\
  \faTwitter ~ \texttt{@renard\_0}\\
  \faGithub ~ \texttt{https://github.com/renard/tikz-penciline}
}
\date{\csname ver@tikz-penciline.sty\endcsname\footnotetext{This
    documentation was compiled on {\platformname} using {\compilator}
    {\cversion}, {\PGF} {\pgfversion}, \texttt{tcolorbox} {\tcbversion} on
    {\today}.}}

\begin{document}

\maketitle


\texttt{tikz-penciline} is based on \textit{percusse} answer from
StackExachange's \textit{simulating hand drawn
  lines}\footnote{http://tex.stackexchange.com/questions/39296}.  The
original idead is to make a \docAuxEnvironment*{tikzpicture} look like if it
was hand drawn.

\begin{tcolorbox}[title={Contents},fonttitle=\bfseries\Large,
  colback=yellow!10!white,colframe=red!50!black,before=\par\bigskip\noindent]
  \makeatletter
  \@starttoc{toc}
  \makeatother
  %\tableofcontents
\end{tcolorbox}


\newpage


\section{Overview}

The original image looks to someting similar to:

%% \begin{exampleB}{}
  \begin{tikzpicture}[]
    \draw[style=help lines] (-2,-2) grid[step=1cm] (4,4);
    \draw[thick] (0,0) -- (0,3) -- (3,3);
     %% This is supposed to be an arc!!
    \draw[ultra thick,blue] (3,3)  arc (0:-90:2cm);
    \draw[thick,pattern=north east lines] (-0.4cm,-0.8cm) rectangle (1.2,-2);
     %% That's not even an ellipse !!
    \node[draw,inner sep=0.5cm,fill=yellow,circle] (a) at (2,0) {};
    \node[draw,inner sep=0.3cm,fill=red] (b) at (2,-2) {};
     %% This was supposed to be an edge!!
    \draw[] (b) to[in=-45,out=45] (a);
    \node[draw,minimum height=2cm,minimum width=1cm] (c) at (-1.5,0) {};
    \draw[->,dashed] (-0.5cm,-0.5cm) -- (-0.5cm,3.5cm)  -| (c.north);
  \end{tikzpicture}
%%\end{exampleB}

The \docAuxEnvironment*{penciline} version looks to someting similar to:

\begin{exampleB}{}
  \begin{tikzpicture}[penciline={jag ratio=1}]
    \draw[decorate,penciline={jag ratio=2},style=help lines] (-2,-2)
    grid[step=1cm] (4,4);
    \draw[decorate,thick] (0,0) -- (0,3) -- (3,3);
     %% This is supposed to be an arc!!
    \draw[decorate,penciline={jag ratio=0,arc angle base2=90},ultra thick,blue] (3,3) arc (0:-90:2cm);
    \draw[decorate,penciline={jag ratio=0},ultra thick,blue] (3,0.5) arc (0:90:2cm);
    \draw[decorate,thick,pattern=north east lines] (-0.4cm,-0.8cm) rectangle (1.2,-2);
     %% That's not even an ellipse !!
    \node[decorate,penciline={jag ratio=0,arc radius base1=0.4,arc radius base2=.6},
      draw,inner sep=0.5cm,fill=yellow,circle] (a) at (2,0) {};
    \node[decorate,draw,inner sep=0.3cm,fill=red] (b) at (2,-2) {};
     %% This was supposed to be an edge!!
    \draw[decorate,penciline={jag ratio=0}] (b) to[in=-45,out=45] (a);
    \node[decorate,draw,minimum height=2cm,minimum width=1cm] (c) at (-1.5,0) {};
    \draw[decorate,->,dashed] (-0.5cm,-0.5cm) -- (-0.5cm,3.5cm)  -| (c.north);
  \end{tikzpicture}
\end{exampleB}


\newpage


\section{Installation}

To install the \docAuxEnvironment*{tkiz-penciline} package copy its directory
to either to
\begin{itemize}
\item \verb"$TEXHOME/tex/latex/"
\item \verb"$TEXMFHOME/tex/latex/"
\item \verb"~/texmf/tex/latex/"
\item \verb"~/Library/texmf/tex/latex/"
\end{itemize}

\section{Usage}

Basically you only have to declare \docAuxEnvironment*{penciline} and
\docAuxEnvironment*{decorate} option to path you want to look hand drawn.


\subsection{customization}

\begin{docKey}{jag ration}{=\meta{float}}{2.0}
  This value controls the \textit{jag} value. The higher the it is, the most
  deformed the drawing will be. Best value are between $2$ and $5$.

  If the path is a circle or an arc you should set \docAuxKey{jag ratio} to
  $0$ for better results.
\end{docKey}


\begin{docKey}{segment x base1}{=\meta{float}}{0.5}
  This is the base for the $x$ coordinate of first control point. The
  formula is given by:

  \textit{length} *
  (\docAuxKey{segment x base1} + \docAuxKey{segment x ratio1} * rnd)

  This means by default that that the $x$ coordinate is about in the middle
  of the segment.
\end{docKey}

\begin{docKey}{segment x ratio1}{=\meta{float}}{0.25}
  This is the randomized ratio part of the $x$ coordinate of first control
  point (see \refKey{segment x base1}).
\end{docKey}

\begin{docKey}{segment x base2}{=\meta{float}}{0.5}
  Base for second control point (see \refKey{segment x base1}).
\end{docKey}

\begin{docKey}{segment x ratio2}{=\meta{float}}{0.25}
  Random ratio for second control point (see \refKey{segment x ratio1}).
\end{docKey}


\begin{docKey}{arc angle base1}{=\meta{float}}{20}
  This is the base of the angle of the polar coordinate for the first
  control point. The formula is given by:

  \docAuxKey{arc angle base1} + rand * \docAuxKey{arc angle ratio1}

  This means that the angle is \docAuxKey{arc angle base1} $\pm$
  \docAuxKey{arc angle ratio1}.
\end{docKey}

\begin{docKey}{arc angle ratio1}{=\meta{float}}{10}
  This is the ratio (variable) part of the angle of the polar coordinate for
  the first control point (see \refKey{arc angle base1}).
\end{docKey}

\begin{docKey}{arc radius base1}{=\meta{float}}{0.3}
  This is the base of the radius of the polar coordinate for the first
  control point. The formula is given by:

  \textit{length} *
  \docAuxKey{arc radius base1} + rand * \docAuxKey{arc radius ratio1}

\end{docKey}

\begin{docKey}{arc radius ratio1}{=\meta{float}}{0.05}
  This is the ratio (variable) part of the radius of the polar coordinate for
  the first control point (see \refKey{arc radius base1}).
\end{docKey}


\begin{docKey}{arc angle base2}{=\meta{float}}{-90}
  Base angle for the second control point. See \refKey{arc angle base1}.
\end{docKey}

\begin{docKey}{arc angle ratio2}{=\meta{float}}{-10}
  Variable part of the angle for the second control point. See \refKey{arc angle ratio1}.
\end{docKey}

\begin{docKey}{arc radius base2}{=\meta{float}}{0.7}
    Base of the radius for the second control point. See \refKey{arc radius base1}.
\end{docKey}

\begin{docKey}{arc radius ratio2}{=\meta{float}}{0.05}
    Variable part of the radius for the second control point. See
    \refKey{arc radius ratio1}.
\end{docKey}


\begin{docKey}{x offset}{=\meta{float}}{1}
  Maximum $x$ offset for target point.
\end{docKey}

\begin{docKey}{y offset}{=\meta{float}}{1}
  Maximum $y$ offset for target point.
\end{docKey}






%% \begin{tikzpicture}[]
%%   \draw[color=red,line width=0.01]
%%   coordinate (A) at (-.25,.25) {} plot[mark=x] (A) node [below] at (A) {\tiny{A}}
%%   coordinate (B) at (3.25,2.75) {} plot[mark=x] (B) node [below] at (B) {\tiny{B}};

%%   \draw
%%   coordinate (sx1min) at ($(A)!\pgfkeysvalueof{/tikz/penciline/segment x base1}!(B)$) {}
%%   coordinate (sx1max) at ($(A)!\pgfkeysvalueof{/tikz/penciline/segment x base1}+\pgfkeysvalueof{/tikz/penciline/segment x ratio1}!(B)$) {}
%%   coordinate (sx2min) at ($(A)!\pgfkeysvalueof{/tikz/penciline/segment x base2}!(B)$) {}
%%   coordinate (sx2max) at ($(A)!\pgfkeysvalueof{/tikz/penciline/segment x base2}+\pgfkeysvalueof{/tikz/penciline/segment x ratio1}!(B)$) {};

  
%%   \draw
%%   coordinate (M1) at ($(sx1min)!\pgfkeysvalueof{/tikz/penciline/jag ratio}*\pgfdecorationsegmentamplitude!90:(A)$) {}
%%   coordinate (M2) at ($(sx1max)!\pgfkeysvalueof{/tikz/penciline/jag ratio}*\pgfdecorationsegmentamplitude!90:(A)$) {}
%%   coordinate (M3) at ($(sx1min)!-\pgfkeysvalueof{/tikz/penciline/jag ratio}*\pgfdecorationsegmentamplitude!90:(A)$) {}
%%   coordinate (M4) at ($(sx1max)!-\pgfkeysvalueof{/tikz/penciline/jag ratio}*\pgfdecorationsegmentamplitude!90:(A)$) {};

%%   \draw
%%   coordinate (N1) at ($(sx2min)!\pgfkeysvalueof{/tikz/penciline/jag ratio}*\pgfdecorationsegmentamplitude!90:(A)$) {}
%%   coordinate (N2) at ($(sx2max)!\pgfkeysvalueof{/tikz/penciline/jag ratio}*\pgfdecorationsegmentamplitude!90:(A)$) {}
%%   coordinate (N3) at ($(sx2min)!-\pgfkeysvalueof{/tikz/penciline/jag ratio}*\pgfdecorationsegmentamplitude!90:(A)$) {}
%%   coordinate (N4) at ($(sx2max)!-\pgfkeysvalueof{/tikz/penciline/jag ratio}*\pgfdecorationsegmentamplitude!90:(A)$) {};

  
%%   \draw[] (M1) -- (M3) -- (M4) -- (M2) -- cycle;
%%   \draw[color=blue] (N1) -- (N3) -- (N4) -- (N2) -- cycle;

%% \end{tikzpicture}




%% \begin{tikzpicture}[]
%%   \draw[color=red,line width=0.01]
%%   coordinate (A) at (-.25,.25) {} plot[mark=x] (A) node [below] at (A) {\tiny{A}}
%%   coordinate (B) at (3.25,2.75) {} plot[mark=x] (B) node [below] at (B) {\tiny{B}};

%%   \draw
%%   coordinate (sx1min) at ($(A)!\pgfkeysvalueof{/tikz/penciline/segment x base1}!(B)$) {}
%%   coordinate (sx1max) at ($(A)!\pgfkeysvalueof{/tikz/penciline/segment x base1}+\pgfkeysvalueof{/tikz/penciline/segment x ratio1}!(B)$) {}
%%   coordinate (sx2min) at ($(A)!\pgfkeysvalueof{/tikz/penciline/segment x base2}!(B)$) {}
%%   coordinate (sx2max) at ($(A)!\pgfkeysvalueof{/tikz/penciline/segment x base2}+\pgfkeysvalueof{/tikz/penciline/segment x ratio1}!(B)$) {};

%%   \draw let \p1 = ($ (A) - (B) $), \n1={veclen(\x1,\y1)} in (A);
%%   \draw[fill=gray!30]  (A) arc
%%   ($\pgfkeysvalueof{/tikz/penciline/arc angle base1}+\pgfkeysvalueof{/tikz/penciline/arc angle ratio1}:\pgfkeysvalueof{/tikz/penciline/arc angle base1}-\pgfkeysvalueof{/tikz/penciline/arc angle ratio1}:\pgfkeysvalueof{/tikz/penciline/arc radius base1}+\pgfkeysvalueof{/tikz/penciline/arc radius ratio1}$);

%% \end{tikzpicture}




\end{document}
